\documentclass{article}

% --- PACKAGES ---
\usepackage{graphicx}
\usepackage{hyperref}
\usepackage[
    a4paper,
    margin=1in
]{geometry}
\usepackage{parskip}
\usepackage{sectsty} % Added to control section formatting

% --- DOCUMENT INFORMATION ---
\title{Autonomous Mapping Robot}
\author{Elman Steve Laguna}
\date{\today}

% --- COMMANDS to prevent page breaks ---
\sectionfont{\nopagebreak}
\subsectionfont{\nopagebreak}
\subsubsectionfont{\nopagebreak}
\widowpenalty=10000
\clubpenalty=10000

\begin{document}

\maketitle


% ===================================================================
% SECTION 1: REPORT ON THE PROBLEM
% ===================================================================
\section{Short Report on Problem}

% --- SUBSECTION: PROJECT DESCRIPTION ---
\subsection{Description of the Project}
This project aims to investigate and implement algorithms for autonomous environment mapping and navigation using a simulated agent. The primary goal is to develop a system that can explore an unknown area, construct a map, and calculate efficient routes to a target without human intervention.

% --- SUBSECTION: PROJECT IDEAS & GOALS ---
\subsection{Project Ideas and Goals}

\subsubsection{Idea 1: 2D Environment Mapping and Pathfinding}
The core problem is to enable an agent to automatically move through and map an unknown 2D area. This investigation will be approached through a comparative study using three distinct simulation environments to understand the trade-offs and challenges of each.
The main deliverable for this idea is a functional simulation where an agent successfully maps its environment and navigates around obstacles.

\subsubsection{Idea 2: 2D Maze Traversal and Optimization (Alternative)}
As an alternative or extension, this idea focuses more specifically on pathfinding optimization within a known environment, such as a maze. Using the same three engines (ROS, PyGame, Bevy), the goal would be to implement and compare various search algorithms (e.g., A*, Dijkstra's) to determine the most efficient path to a target—"the fastest to the cheese." This shifts the focus from exploration and mapping (SLAM) to pure algorithmic performance.

% --- SUBSECTION: LITERATURE SEARCH ---
\subsection{Preliminary Literature Search}
Initial research focuses on \textbf{Simultaneous Localization and Mapping (SLAM)}, the core problem of concurrently building a map while tracking an agent's position within it. To simulate sensor data for perception, the project will implement \textbf{ray casting}, using line-intersection calculations to mimic how a Lidar scanner detects obstacles and builds a map. \textbf{Procedural maze generation} will be used to create complex, structured environments for testing the navigation logic. Finally, once a map is generated, \textbf{shortest path algorithms} such as Dijkstra's and A* Search will be investigated and implemented to provide the agent with efficient navigation capabilities. Initially, I will start with a simple handmaid map. Them make the tasks incrementally more difficult.

% --- SUBSECTION: PROPOSED APPROACH ---
\subsection{Ideas on Approach}
The proposed approach will begin with the 2D Environment Mapping idea. The project will be executed in three phases:
\begin{enumerate}
    \item \textbf{Phase 1: Foundational Learning with ROS.} I will start by mastering the basics of agent control and sensing within a structured environment using the ROS Turtlesim package. This will establish a baseline understanding of robotic simulation.
    \item \textbf{Phase 2: Custom Simulation with PyGame.} Next, I will develop a 2D simulation from scratch in Python with PyGame. This will provide the flexibility to implement and visualize a mapping algorithm (such as a basic grid-based or occupancy grid map) and a pathfinding algorithm (like A*).
    \item \textbf{Phase 3: Advanced Implementation in Bevy.} Finally, I will replicate the simulation in Rust using the Bevy engine. This phase will focus on evaluating the performance, modularity, and scalability benefits of a modern game engine for robotics simulations.
\end{enumerate}
Throughout these phases, the primary challenge will be to process simulated sensor data (e.g., raycasting to simulate Lidar) to build the map and inform the navigation logic.



\newpage
\subsection*{Small Maze Generation}

\begin{figure}[h!]
    \centering
    \includegraphics[width=0.8\textwidth]{small_map.png}
    \caption{A 5$\times$5 maze configuration generating an 11$\times$11 tile grid. 
    Final dimensions calculated as: (\texttt{MAZE\_WIDTH} $\times$ 2 + 1) $\times$ (\texttt{MAZE\_HEIGHT} $\times$ 2 + 1)}
    \label{fig:small_maze}
\end{figure}

\newpage
\subsection*{Medium Maze Generation}

\begin{figure}[h!]
    \centering
    \includegraphics[width=0.8\textwidth]{medium_map.png}
    \caption{A 10$\times$10 maze configuration generating a 21$\times$21 tile grid. 
    Final dimensions calculated as: (\texttt{MAZE\_WIDTH} $\times$ 2 + 1) $\times$ (\texttt{MAZE\_HEIGHT} $\times$ 2 + 1)}
    \label{fig:medium_maze}
\end{figure}
    Final dimensions calculated as: (\texttt{MAZE\_WIDTH} $\times$ 2 + 1) $\times$ (\texttt{MAZE\_HEIGHT} $\times$ 2 + 1)}
    \label{fig:small_maze}
\end{figure}

\newpage
\subsection*{Large Maze Generation}

\begin{figure}[h!]
    \centering
    \includegraphics[width=0.8\textwidth]{large_map.png}
    \caption{A 50$\times$50 maze configuration generating a 101$\times$101 tile grid. 
    Final dimensions calculated as: (\texttt{MAZE\_WIDTH} $\times$ 2 + 1) $\times$ (\texttt{MAZE\_HEIGHT} $\times$ 2 + 1)}
    \label{fig:large_maze}
\end{figure}
    Final dimensions calculated as: (\texttt{MAZE\_WIDTH} $\times$ 2 + 1) $\times$ (\texttt{MAZE\_HEIGHT} $\times$ 2 + 1)}
    \label{fig:small_maze}
\end{figure}



\begin{thebibliography}{9}

% --- Academic Paper ---
\bibitem{slam_review}
M. Aborizky, H. A. As’ari, A. Fadlil and R. D. P. W. "Lidar-Based 2D SLAM for Mobile Robot in an Indoor Environment: A Review," \textit{2021 9th International Conference on Information and Communication Technology (ICoICT)}, 2021, pp. 488-493. doi: 10.1109/ICoICT52021.2021.9538731.

% --- Web Articles & Tutorials ---
\bibitem{raycasting_tutorial}
L. Vandevenne, "Raycasting," \textit{Lode's Computer Graphics Tutorial}. [Online]. Available: \url{https://lodev.org/cgtutor/raycasting.html}

\bibitem{lidar_basics}
NEON Science, "Lidar Basics," \textit{National Ecological Observatory Network}, 2021. [Online]. Available: \url{https://www.neonscience.org/resources/learning-hub/tutorials/lidar-basics}

\bibitem{line_intersection}
Zona Land Education, "Intersection of Two Lines," \textit{Zona Land Education}. [Online]. Available: \url{http://zonalandeducation.com/mmts/intersections/intersectionOfTwoLines1/intersectionOfTwoLines1.html}

% --- Videos ---
\bibitem{lidar_video}
NEON Science, "How Does LiDAR Remote Sensing Work? Light Detection and Ranging," \textit{YouTube}, Nov. 24, 2014. [Video]. Available: \url{https://www.youtube.com/watch?v=EYbhNSUnIdU}

\bibitem{pygame_raycast_video}
FinFET, "Raycasting with Pygame in Python! Simple 3D game tutorial Devlog," \textit{YouTube}, Nov. 6, 2021. [Video]. Available: \url{https://www.youtube.com/watch?v=4gqPv7A_YRY}

\bibitem{raycast_explained_video}
WeirdDevers, "Ray casting fully explained. Pseudo 3D game," \textit{YouTube}, Dec. 27, 2020. [Video]. Available: \url{https://www.youtube.com/watch?v=g8p7nAbDz6Y}

\bibitem{maze_solving_competition_video}
Veritasium, "The Fastest Maze-Solving Competition On Earth," \textit{YouTube}, May 24, 2023. [Video]. Available: \url{https://www.youtube.com/watch?v=ZMQbHMgK2rw}

\bibitem{mazes_school_video}
mattbatwings, "What School Didn't Tell You About Mazes \#SoMEpi," \textit{YouTube}, Jun. 22, 2024. [Video]. Available: \url{https://www.youtube.com/watch?v=uctN47p_KVk}

\bibitem{slam_robot_video}
Kai Nakamura, "How to Make an Autonomous Mapping Robot Using SLAM," \textit{YouTube}, May 14, 2024. [Video]. Available: \url{https://www.youtube.com/watch?v=xqjVTE7QvOg}

\bibitem{lidar_ros_video}
Articulated Robotics, "How do we add LIDAR to a ROS robot?," \textit{YouTube}, Jun. 2, 2022. [Video]. Available: \url{https://www.youtube.com/watch?v=eJZXRncGaGM}

\end{thebibliography}

\end{document}